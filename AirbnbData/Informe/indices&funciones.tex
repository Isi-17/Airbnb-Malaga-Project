\subsection{Índices}

La creación de índices es una estrategia importante para mejorar la velocidad de búsqueda en bases de datos. A continuación, se justifica la creación de índices para cada uno de los siguientes atributos en las tablas correspondientes:

\subsection*{Tabla ``listings''}

\begin{itemize}
  \item \textbf{Atributo "id":} Se ha creado el índice \texttt{idx\_listings\_id} en el atributo "id" de la tabla "listings" para acelerar las búsquedas por este identificador único.
  \item \textbf{Atributo "host\_id":} Se ha creado el índice \texttt{idx\_listings\_host\_id} en el atributo "host\_id" de la tabla "listings" para mejorar la velocidad de búsqueda de registros relacionados con un host específico.
  \item \textbf{Atributos "host\_id" y "property\_type":} El índice \\ \texttt{idx\_listings\_host\_id\_property\_type} en los atributos "host\_id" y "property\_type" permite realizar consultas más eficientes que involucren la combinación de estos dos atributos.
  \item \textbf{Atributos de ubicación:} Se ha creado el índice \texttt{idx\_listings\_location} utilizando el método de índice espacial \texttt{gist} en los atributos de latitud y longitud. Esto mejora la velocidad de búsqueda de registros basados en la ubicación geográfica.
  \item \textbf{Atributos de distancia:} El índice \texttt{idx\_listings\_distance} se ha creado utilizando una expresión de cálculo de distancia. Esto mejora la eficiencia en las búsquedas que involucren cálculos de distancia utilizando la fórmula proporcionada.
\end{itemize}

\subsection*{Tabla ``reviews''}

\begin{itemize}
  \item \textbf{Atributo "listing\_id":} Se ha creado el índice \texttt{idx\_reviews\_listing\_id} en el atributo "listing\_id" de la tabla "reviews" para acelerar las búsquedas por el identificador de las propiedades asociadas a las revisiones.
  \item \textbf{Atributo "comments":} El índice \texttt{idx\_listings\_review} se ha creado utilizando el método \texttt{gin} para la creación de índices de texto completo en el atributo "comments". Esto permite realizar búsquedas de texto en el campo de comentarios de manera más rápida y eficiente.
\end{itemize}

\subsection*{Tabla ``calendar''}


\begin{itemize}
  \item \textbf{Atributos "listing\_id" y "date":} El índice \\ \texttt{idx\_calendar\_listing\_id\_date} en los atributos "listing\_id" y "date" de la tabla "calendar" mejora la velocidad de búsqueda en consultas que involucren estos dos atributos de manera conjunta.
  \item \textbf{Atributo "available":} Se ha creado el índice \texttt{idx\_calendar\_availability} en el atributo "available" para acelerar las búsquedas de disponibilidad en el calendario.
  \item \textbf{Atributo "listing\_id":} El índice \texttt{idx\_calendar\_listing\_id} en el atributo "listing\_id" mejora la eficiencia en búsquedas relacionadas con el identificador de las propiedades en el calendario.
\end{itemize}

\subsection{Creación de Vistas}

Se crean varias vistas en la base de datos para facilitar el análisis y la visualización de los datos. Estas vistas incluyen:

\begin{itemize}
    \item \texttt{year\_availability}: Muestra los alojamientos con mayor disponibilidad anual.
    \item \texttt{listing\_availability\_price}: Muestra la disponibilidad y el precio diario de los listados.
    \item \texttt{price\_by\_bedrooms}: Muestra el precio medio según el número de habitaciones.
    \item \texttt{price\_trends}: Muestra las tendencias de precios medios a lo largo de los años.
    \item \texttt{availability\_trends}: Muestra las tendencias de disponibilidad a lo largo de los años.
    \item \texttt{top\_rated\_listings}: Muestra los alojamientos mejor valorados por categoría de propiedad.
    \item \texttt{negative\_reviews}: Muestra los alojamientos con comentarios negativos.
    \item \texttt{owner\_negative\_reviews}: Muestra los propietarios con más reseñas con comentarios negativos.
\end{itemize}
\newpage
\subsection{Funciones}

El código también incluye la definición de funciones que realizan diferentes cálculos y facilitan el posterior análisis de los datos. Estas funciones incluyen:

\begin{itemize}
    \item \texttt{calculate\_average\_rating()}: Calcula el promedio de las calificaciones de los alojamientos.
\end{itemize}
    \small 
    \begin{verbatim}
CREATE OR REPLACE FUNCTION calculate_average_rating() 
    RETURNS FLOAT AS $$
DECLARE
  avg_rating FLOAT;
BEGIN
  SELECT AVG(review_scores_rating) INTO avg_rating
  FROM listings;
  RETURN avg_rating;
END;
$$ LANGUAGE plpgsql;
    \end{verbatim}
\begin{itemize}
    \item \texttt{calculate\_max\_minimum\_nights()}: Calcula el número mínimo de noches para alojarse en los listados.
\end{itemize}
    \small 
    \begin{verbatim}
CREATE OR REPLACE FUNCTION calculate_max_minimum_nights()
    RETURNS INTEGER AS $$
DECLARE
    max_min_nights INTEGER;
BEGIN
    SELECT MAX(minimum_nights) INTO max_min_nights
    FROM calendar
    GROUP BY listing_id
    ORDER BY max_min_nights DESC
    LIMIT 1;
    RETURN max_min_nights;
END;
$$ LANGUAGE plpgsql;
    \end{verbatim}
\begin{itemize}
    \item \texttt{calculate\_multiple\_properties\_percentage()}: Calcula el porcentaje de propietarios con más de una propiedad.
\end{itemize}
    \small 
    \begin{verbatim}
CREATE OR REPLACE FUNCTION calculate_multiple_properties_percentage()
    RETURNS FLOAT AS $$
DECLARE
    total_hosts INTEGER;
    hosts_with_multiple_properties INTEGER;
    percentage FLOAT;
BEGIN
    SELECT COUNT(DISTINCT host_id) INTO total_hosts
    FROM listings;

    SELECT COUNT(*) INTO hosts_with_multiple_properties
    FROM (
    SELECT host_id
    FROM listings
    GROUP BY host_id
    HAVING COUNT(*) > 1
    ) AS multiple_properties;

    percentage := (hosts_with_multiple_properties::FLOAT / total_hosts::FLOAT) * 100;
    RETURN ROUND(percentage);
END;
$$ LANGUAGE plpgsql;
    \end{verbatim}

\newpage
